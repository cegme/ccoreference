\documentclass[12pt]{article}
\usepackage{url,graphicx,tabularx,array,geometry}
\setlength{\parskip}{1ex} %--skip lines between paragraphs
\setlength{\parindent}{0pt} %--don't indent paragraphs
%-- Commands for header
\renewcommand{\title}[1]{\textbf{#1}\\}
\renewcommand{\line}{\begin{tabularx}{\textwidth}{X>{\raggedleft}X}\hline\\\end{tabularx}\\[-0.5cm]}
\newcommand{\leftright}[2]{\begin{tabularx}{\textwidth}{X>{\raggedleft}X}#1%
& #2\\\end{tabularx}\\[-0.5cm]}

%\linespread{2} %-- Uncomment for Double Space

\begin{document}
\title{Paper Title}
\line
\leftright{\today}{John Doe} %-- left and right positions in the header

\section{Bishop 9.1}
K-means Clustering objective function, also called a distortion meausre, given by
\begin{equation}
J=\sum_{n=1}^{N}_{k=1}^{K}r_n_k\lVert X_n - u_k\rVert
\end{equation}

\begin{itemize}
\item Have your work neatly typeset
\item Impress your professors who might recognize the fonts
\item Include equations easily
\item Put math $\in$ the text: $\textrm{money} = \sqrt{\textrm{all evil}}$
\item Not ever have to mess with MS Word
\end{itemize}



Being able to include equations is especially nice.  Say you were doing a
homework the Pythagorean theorem:
\begin{equation}
d = \sqrt{a^2+b^2}.
\end{equation}
It is simple to include it right in the text!  In fact, when doing math
homeworks I often find it simpler to code derivations in \LaTeX.  If I make a
mistake somewhere, I can quickly change all down-line equations with a
find-replace procedure instead of having to re-write or cross out large blocks
of text.
\section{Getting Started}
\end{document}
