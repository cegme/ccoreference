\documentclass[12pt,letterpaper,boxed]{amspset}

% include this if you want to import graphics files with /includegraphics
\usepackage{graphicx}

% info for header block in upper right hand corner

\name{Ima G. Student}
\class{AMS 550.671 - Combinatorial Analysis}
\assignment{Homework \#1}
\duedate{09/19/2005}

\begin{document}

\problemlist{K. Bogart	1.1.1, 1.1.2, 1.1.8, 1.1.9, 1.1.13, 1.1.14, 1.1.16, 1.1.18,  \\
					1.2.1, 1.2.2, 1.2.8, 1.2.10, 1.2.12, 1.2.14, 1.2.15, 1.2.20}

\begin{problem}[1.1.1]
A city council with seven members must elect a mayor and vice-mayor
from among its members. In how many ways can the council choose these
officers?
\end{problem}

\begin{solution}

\end{solution}

%%%


\begin{problem}[1.1.2]
In how many ways can you distribute three distinct pieces of fruit
among five children if no child can receive more than one? If any
child can receive any number?
\end{problem}

\begin{solution}

\end{solution}

%%%


\begin{problem}[1.1.8]
A coffee machine allows you to choose your coffee either plain or with
single or double portions of sugar and/or cream. In how many ways can
you choose your coffee?
\end{problem}

\begin{solution}

\end{solution}

%%%


\begin{problem}[1.1.9]
What is the number of 10 digit numbers with no two successive digits
equal? What is the number that have at least one pair of successive
digits equal.
\end{problem}

\begin{solution}

\end{solution}

%%%


\begin{problem}[1.1.13]
A professor has six test questions. Three of them form a unit and must
be kept together in the order the professor has chosen. In how many
ways can the professor arrange the test questions?
\end{problem}

\begin{solution}

\end{solution}

%%%


\begin{problem}[1.1.14]
Answer the equation of Exercise 13 if the three questions that are
kept together can be arranged in any order.
\end{problem}

\begin{solution}

\end{solution}

%%%


\begin{problem}[1.1.16]
Use $Y - X$ to denote the set of elements of $Y$ not in $X$. If every
element of the set $X$ is an element of the set $Y$, apply the sum
principle to the sets $X$ and $Y - X$ to explain why
\[
       | Y - X| = |Y| - |X|.
\]
\end{problem}

\begin{solution}

\end{solution}

%%%


\begin{problem}[1.1.18]
Stirling showed that $n!$ is approximately $\sqrt{2 \pi n} n^{n}
e^{-n}$. Using a computer or scientific calculator, determine the ratio
of $n!$ to this approximation for values of $n$ equal to multiples of
10 up to 50. For each Stirling approximation, multiply it by $1 + 
\frac{1}{12n}$ and find the ratio of the result to $n!$.
\end{problem}

\begin{solution}

\end{solution}

%%%


\begin{problem}[1.2.1]
Write down all the sets of ordered pairs that correspond to possible
functions from $\{ 0, 1 \}$ to $\{1, 2, 3 \}$.
\end{problem}

\begin{solution}

\end{solution}

%%%


\begin{problem}[1.2.2]
Write down all the sets of ordered pairs that correspond to possible
functions from $\{ 1, 2, 3 \}$ to $\{0, 1 \}$.
\end{problem}

\begin{solution}

\end{solution}

%%%


\begin{problem}[1.2.8]
Which of the following relations are functions? If not, why not?
\begin{itemize}
  \item{(a)}      $\{ (a,1), (b,2), (c,1)\}$
  \item{(b)}      $\{ (1,a), (2,b), (3,b), (1,c)\}$
  \item{(c)}      $\{ (-1,1), (0,0), (1,1), (2,4), (-2,4)\}$
  \item{(d)}      $\{ (1,-1), (0,0), (1,1), (4,2), (4,-2) \}$
  \item{(e)}      $\{ (0,0), (1,1), (4,2)\}$
\end{itemize}
\end{problem}

\begin{solution}
\begin{itemize}
  \item{(a)}      
  \item{(b)}      
  \item{(c)}      
  \item{(d)}      
  \item{(e)}      
\end{itemize}

\end{solution}

%%%


\begin{problem}[1.2.10]
How many functions from a four-element set to a five-element set are
not one-to-one? How many functions from a five-element set to a
four-element set are not one-to-one?
\end{problem}

\begin{solution}

\end{solution}

%%%


\begin{problem}[1.2.12]
A group organizing a faculty-student tennis match must match five
faculty volunteers with five of the 12 students who volunteered to be
in the match. In how many ways can they do this?
\end{problem}

\begin{solution}

\end{solution}

%%%


\begin{problem}[1.2.14]
\begin{itemize}
  \item{(a)}      In how many ways can you pass out 10 different pieces of
candy to three children?
  \item{(b)}      What if each child must get at least one piece? (Hint: To
answer this question, ask yourself in how many distributions does one
child get candy? In how many distributions do exactly two children get
candy?)
\end{itemize}
\end{problem}
\begin{itemize}
  \item{(a)}      
  \item{(b)}      
\end{itemize}

\begin{solution}

\end{solution}

%%%


\begin{problem}[1.2.15 (optional)]
The inverse of a relation $R$, denoted by $R^{-1}$, is the set of all
ordered pairs $(y,x)$ such that $(x,y)$ is in $R$. Show that the
inverse relation of a function $f$ is a function (whose domain is some
subset of the range of $f$) if and only if $f$ is one-to-one. Show
that the inverse relation of a function $f$ is a function whose domain
is the range of $f$ if and only if $f$ is a bijection.
\end{problem}

\begin{solution}

\end{solution}

%%%


\begin{problem}[1.2.20 (optional)]
Show that if $S$ and $T$ are of the same size, then a function $f: S
\rightarrow T$ is onto if and only if it is one-to-one. Discuss what
this means about testing a function between two sets of the same size
to see if it is a bijection.
\end{problem}

\begin{solution}

\end{solution}




\end{document}

